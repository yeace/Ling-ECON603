%Input preamble
%Style
\documentclass[12pt]{article}
\usepackage[top=1in, bottom=1in, left=1in, right=1in]{geometry}
\parindent 22pt
\usepackage{fancyhdr}

%Packages
\usepackage{adjustbox}
\usepackage{amsmath}
\usepackage{amsfonts}
\usepackage{amssymb}
\usepackage[english]{babel}
\usepackage{bm}
\usepackage[table]{xcolor}
\usepackage{tabu}
\usepackage{color,soul}
\usepackage[utf8x]{inputenc}
\usepackage{makecell}
\usepackage{longtable}
\usepackage{multirow}
\usepackage[normalem]{ulem}
\usepackage{etoolbox}
\usepackage{graphicx}
\usepackage{tabularx}
\usepackage{ragged2e}
\usepackage{booktabs}
\usepackage{caption}
\usepackage{fixltx2e}
\usepackage[para, flushleft]{threeparttablex}
\usepackage[capposition=top]{floatrow}
\usepackage{subcaption}
\usepackage{pdfpages}
\usepackage{pdflscape}
\usepackage[sort&compress]{natbib}
\usepackage{bibunits}
\usepackage[colorlinks=true,linkcolor=darkgray,citecolor=darkgray,urlcolor=darkgray,anchorcolor=darkgray]{hyperref}
\usepackage{marvosym}
\usepackage{makeidx}
\usepackage{setspace}
\usepackage{enumerate}
\usepackage{rotating}
\usepackage{epstopdf}
\usepackage[titletoc]{appendix}
\usepackage{framed}
\usepackage{comment}
\usepackage{xr}
\usepackage{titlesec}
\usepackage{footnote}
\usepackage{longtable}
\newlength{\tablewidth}
\setlength{\tablewidth}{9.3in}
\usepackage[bottom]{footmisc}
\usepackage{stackengine}
\newcommand\barbelow[1]{\stackunder[1.2pt]{$#1$}{\rule{1ex}{.085ex}}}
\usepackage{titletoc}
\usepackage{accents}
\usepackage{arydshln }
\usepackage{titletoc}
\titlespacing{\section}{.2pt}{1ex}{1ex}
\setcounter{section}{0}
\renewcommand{\thesection}{\arabic{section}}


\makeatletter
\pretocmd\start@align
{%
  \let\everycr\CT@everycr
  \CT@start
}{}{}
\apptocmd{\endalign}{\CT@end}{}{}
\makeatother
%Watermark
\usepackage[printwatermark]{xwatermark}
\usepackage{lipsum}
\definecolor{lightgray}{RGB}{220,220,220}
\definecolor{dimgray}{RGB}{105,105,105}

%\newwatermark[allpages,color=lightgray,angle=45,scale=3,xpos=0,ypos=0]{Preliminary Draft}

%Further subsection level
\usepackage{titlesec}
\titleformat{\paragraph}
{\normalfont\normalsize\bfseries}{\theparagraph}{1em}{}
\titlespacing*{\paragraph}
{0pt}{3.25ex plus 1ex minus .2ex}{1.5ex plus .2ex}

\titleformat{\subparagraph}
{\normalfont\normalsize\bfseries}{\thesubparagraph}{1em}{}
\titlespacing*{\subparagraph}
{0pt}{3.25ex plus 1ex minus .2ex}{1.5ex plus .2ex}

%Functions
\DeclareMathOperator{\cov}{Cov}
\DeclareMathOperator{\sign}{sgn}
\DeclareMathOperator{\var}{Var}
\DeclareMathOperator{\plim}{plim}
\DeclareMathOperator*{\argmin}{arg\,min}
\DeclareMathOperator*{\argmax}{arg\,max}

%Math Environments
\usepackage{amsthm}
\newtheoremstyle{mytheoremstyle} % name
    {\topsep}                    % Space above
    {\topsep}                    % Space below
    {\color{black}}                   % Body font
    {}                           % Indent amount
    {\itshape \color{dimgray}}                   % Theorem head font
    {.}                          % Punctuation after theorem head
    {.5em}                       % Space after theorem head
    {}  % Theorem head spec (can be left empty, meaning ?normal?)

\theoremstyle{mytheoremstyle}
\newtheorem{assumption}{Assumption}
\renewcommand\theassumption{\arabic{assumption}}

\theoremstyle{mytheoremstyle}
\newtheorem{assumptiona}{Assumption}
\renewcommand\theassumptiona{\arabic{assumptiona}a}

\newtheorem{assumptionb}{Assumption}
\renewcommand\theassumptionb{\arabic{assumptionb}b}

\newtheorem{assumptionc}{Assumption}
\renewcommand\theassumptionc{\arabic{assumptionc}c}

\theoremstyle{mytheoremstyle}
\newtheorem{lemma}{Lemma}

\theoremstyle{mytheoremstyle}
\newtheorem{proposition}{Proposition}

\theoremstyle{mytheoremstyle}
\newtheorem{corollary}{Corollary}

%Commands
\newcommand\independent{\protect\mathpalette{\protect\independenT}{\perp}}
\def\independenT#1#2{\mathrel{\rlap{$#1#2$}\mkern2mu{#1#2}}}
\newcommand{\overbar}[1]{\mkern 1.5mu\overline{\mkern-1.5mu#1\mkern-1.5mu}\mkern 1.5mu}
\newcommand{\equald}{\ensuremath{\overset{d}{=}}}
\captionsetup[table]{skip=10pt}
%\makeindex

%Table, Figure, and Section Styles
\captionsetup[figure]{labelfont={bf},name={Figure},labelsep=period}
\renewcommand{\thefigure}{\arabic{figure}}
\captionsetup[table]{labelfont={bf},name={Table},labelsep=period}
\renewcommand{\thetable}{\arabic{table}}
\titleformat{\section}{\centering \normalsize \bf}{\thesection.}{0em}{}%\titlespacing*{\subsection}{0pt}{0\baselineskip}{0\baselineskip}
\renewcommand{\thesection}{\arabic{section}}

\titleformat{\subsection}{\flushleft \normalsize \bf}{\thesubsection}{0em}{}
\renewcommand{\thesubsection}{\arabic{section}.\arabic{subsection}}

%No indent
\setlength\parindent{24pt}
\setlength{\parskip}{5pt}

%Logo
%\AddToShipoutPictureBG{%
%  \AtPageUpperLeft{\raisebox{-\height}{\includegraphics[width=1.5cm]{uchicago.png}}}
%}

\newcolumntype{L}[1]{>{\raggedright\let\newline\\\arraybackslash\hspace{0pt}}m{#1}}
\newcolumntype{C}[1]{>{\centering\let\newline\\\arraybackslash\hspace{0pt}}m{#1}}
\newcolumntype{R}[1]{>{\raggedleft\let\newline\\\arraybackslash\hspace{0pt}}m{#1}} 

\newcommand{\mr}{\multirow}
\newcommand{\mc}{\multicolumn}

%\newcommand{\comment}[1]{}

\begin{document}


\title{\Large \textbf{Does Entrepreneurship Mitigate or Exacerbate Social Mobility? }}
\author{Yi Ling}

\date{\today}

\maketitle

\thispagestyle{empty} 
\doublespacing
\thispagestyle{empty} 

\section{Research Question}
This proposal aims to compare the outcomes of two types of entrepreneurs:  first-generation entrepreneurs and entrepreneurs whose family is also self-employed, and examine whether the first type faces lower intergenerational mobility than the second type, in other words, can people from a non-entrepreneur family background climb up the social ladder through choosing to start a business?

\section{Literature Review}
This research explores the link between entrepreneurship and intergenerational mobility, whether choosing to be an entrepreneur will increase intergenerational mobility. The first relates to the entrepreneurship's impact on wealth concentration and inequality. \citet{quadrini1999importance} points out that entrepreneurship has implications for high wealth concentration: entrepreneurs have higher asset holdings even though their income is not larger than non-entrepreneurs. Moreover, entrepreneurs have greater upward mobility, which means they have a greater probability of moving to higher wealth classes, and this is not only due to their higher incomes. \citet{cagetti2006entrepreneurship} also emphasizes that the reason why entrepreneurs have higher wealth is because they save more for borrowing constraints, and share the fortune with their children to keep the family firm working. This research aims to find whether the saving patterns of two types of entrepreneurs are different, which leads to the high saving rate of entrepreneurs.


The second relates to the literature on social mobility. Intergenerational mobility has long been a central topic of interest for economists worldwide, as greater inequality in a society tends to make family background a stronger determinant of offspring’s outcomes \citep{corakIncomeInequalityEquality2013a}. There are many discussions on how parental self-employment impacts children's occupation decisions \citep{dunn2000financial, laferrere2001self, djankov2006china, barnir2011parental, lindquist2015entrepreneurial}. This research is most closely related to \citet{barnir2011parental, lindquist2015entrepreneurial}. \citet{barnir2011parental} compare first-generation self-employers and self-employers who have entrepreneurial parents, they found that through vicarious learning, people who have entrepreneurial parents can observe and learn from their family, which makes it easier to become an entrepreneur, but they face the same financial constraints when their firm enters the market.  \citet{lindquist2015entrepreneurial} also compares the difference between the entrepreneurs whose parents were self-employed and those whose parents were not, they found that parental entrepreneurship increases the probability of children’s entrepreneurship by about 60\% due to role modeling.  Other papers consider the occupation decision within the entrepreneur family. \citet{dunn2000financial, laferrere2001self, djankov2006china} found offspring of the self-employed display a greater propensity to become entrepreneurs. Previous literature also explores the intergenerational mobility in many other aspects \citep{peters1992patterns, checchi1999more}. For instance, \citet{chetty2018impacts} analyzes the neighborhood's impact on intergenerational mobility, finding out the outcome of the children is better when they spend their childhood in better opportunity areas.


The third relates to the macroeconomic and public literature on determinants of entrepreneurship and entrepreneurial returns. Since the returns of two types of entrepreneurs might be different, this might be the reason why entrepreneurial returns are lower than workers. The classic puzzle in macroeconomics is the "Private Equity Premium Puzzle", which means entrepreneurs have lower returns than public equity returns \citep{moskowitz2002returns, bhandari2020survey}. Since they have lower returns, the reasons why people choose to be entrepreneurs are widely discussed \citep{parker2005economics, vereshchagina2009risk}. \citet{parker2005economics} summarizes these reasons and adds other explanations: the risk of entrepreneurs' net worth is less than the risk of business, and entrepreneurs could tolerate the high risk in their businesses. Additionally, \citet{vereshchagina2009risk} suggested that non-concavity areas created in the interaction with an occupational choice between being an entrepreneur and an employee could explain the high risk in investment. This research also aims to calculate the returns of two types of entrepreneurial investment. If first-generation entrepreneurs earn lower returns while those with parental self-employment backgrounds earn higher returns, then the apparent puzzle can be explained by the higher failure rate among first-generation entrepreneurs. 

\bibliographystyle{apalike} 
\bibliography{references}

\end{document}

