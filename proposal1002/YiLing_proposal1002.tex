%Input preamble
%Style
\documentclass[12pt]{article}
\usepackage[top=1in, bottom=1in, left=1in, right=1in]{geometry}
\parindent 22pt
\usepackage{fancyhdr}

%Packages
\usepackage{adjustbox}
\usepackage{amsmath}
\usepackage{amsfonts}
\usepackage{amssymb}
\usepackage[english]{babel}
\usepackage{bm}
\usepackage[table]{xcolor}
\usepackage{tabu}
\usepackage{color,soul}
\usepackage[utf8x]{inputenc}
\usepackage{makecell}
\usepackage{longtable}
\usepackage{multirow}
\usepackage[normalem]{ulem}
\usepackage{etoolbox}
\usepackage{graphicx}
\usepackage{tabularx}
\usepackage{ragged2e}
\usepackage{booktabs}
\usepackage{caption}
\usepackage{fixltx2e}
\usepackage[para, flushleft]{threeparttablex}
\usepackage[capposition=top]{floatrow}
\usepackage{subcaption}
\usepackage{pdfpages}
\usepackage{pdflscape}
\usepackage[sort&compress]{natbib}
\usepackage{bibunits}
\usepackage[colorlinks=true,linkcolor=darkgray,citecolor=darkgray,urlcolor=darkgray,anchorcolor=darkgray]{hyperref}
\usepackage{marvosym}
\usepackage{makeidx}
\usepackage{setspace}
\usepackage{enumerate}
\usepackage{rotating}
\usepackage{epstopdf}
\usepackage[titletoc]{appendix}
\usepackage{framed}
\usepackage{comment}
\usepackage{xr}
\usepackage{titlesec}
\usepackage{footnote}
\usepackage{longtable}
\newlength{\tablewidth}
\setlength{\tablewidth}{9.3in}
\usepackage[bottom]{footmisc}
\usepackage{stackengine}
\newcommand\barbelow[1]{\stackunder[1.2pt]{$#1$}{\rule{1ex}{.085ex}}}
\usepackage{titletoc}
\usepackage{accents}
\usepackage{arydshln }
\usepackage{titletoc}
\titlespacing{\section}{.2pt}{1ex}{1ex}
\setcounter{section}{0}
\renewcommand{\thesection}{\arabic{section}}


\makeatletter
\pretocmd\start@align
{%
  \let\everycr\CT@everycr
  \CT@start
}{}{}
\apptocmd{\endalign}{\CT@end}{}{}
\makeatother
%Watermark
\usepackage[printwatermark]{xwatermark}
\usepackage{lipsum}
\definecolor{lightgray}{RGB}{220,220,220}
\definecolor{dimgray}{RGB}{105,105,105}

%\newwatermark[allpages,color=lightgray,angle=45,scale=3,xpos=0,ypos=0]{Preliminary Draft}

%Further subsection level
\usepackage{titlesec}
\titleformat{\paragraph}
{\normalfont\normalsize\bfseries}{\theparagraph}{1em}{}
\titlespacing*{\paragraph}
{0pt}{3.25ex plus 1ex minus .2ex}{1.5ex plus .2ex}

\titleformat{\subparagraph}
{\normalfont\normalsize\bfseries}{\thesubparagraph}{1em}{}
\titlespacing*{\subparagraph}
{0pt}{3.25ex plus 1ex minus .2ex}{1.5ex plus .2ex}

%Functions
\DeclareMathOperator{\cov}{Cov}
\DeclareMathOperator{\sign}{sgn}
\DeclareMathOperator{\var}{Var}
\DeclareMathOperator{\plim}{plim}
\DeclareMathOperator*{\argmin}{arg\,min}
\DeclareMathOperator*{\argmax}{arg\,max}

%Math Environments
\usepackage{amsthm}
\newtheoremstyle{mytheoremstyle} % name
    {\topsep}                    % Space above
    {\topsep}                    % Space below
    {\color{black}}                   % Body font
    {}                           % Indent amount
    {\itshape \color{dimgray}}                   % Theorem head font
    {.}                          % Punctuation after theorem head
    {.5em}                       % Space after theorem head
    {}  % Theorem head spec (can be left empty, meaning ?normal?)

\theoremstyle{mytheoremstyle}
\newtheorem{assumption}{Assumption}
\renewcommand\theassumption{\arabic{assumption}}

\theoremstyle{mytheoremstyle}
\newtheorem{assumptiona}{Assumption}
\renewcommand\theassumptiona{\arabic{assumptiona}a}

\newtheorem{assumptionb}{Assumption}
\renewcommand\theassumptionb{\arabic{assumptionb}b}

\newtheorem{assumptionc}{Assumption}
\renewcommand\theassumptionc{\arabic{assumptionc}c}

\theoremstyle{mytheoremstyle}
\newtheorem{lemma}{Lemma}

\theoremstyle{mytheoremstyle}
\newtheorem{proposition}{Proposition}

\theoremstyle{mytheoremstyle}
\newtheorem{corollary}{Corollary}

%Commands
\newcommand\independent{\protect\mathpalette{\protect\independenT}{\perp}}
\def\independenT#1#2{\mathrel{\rlap{$#1#2$}\mkern2mu{#1#2}}}
\newcommand{\overbar}[1]{\mkern 1.5mu\overline{\mkern-1.5mu#1\mkern-1.5mu}\mkern 1.5mu}
\newcommand{\equald}{\ensuremath{\overset{d}{=}}}
\captionsetup[table]{skip=10pt}
%\makeindex

%Table, Figure, and Section Styles
\captionsetup[figure]{labelfont={bf},name={Figure},labelsep=period}
\renewcommand{\thefigure}{\arabic{figure}}
\captionsetup[table]{labelfont={bf},name={Table},labelsep=period}
\renewcommand{\thetable}{\arabic{table}}
\titleformat{\section}{\centering \normalsize \bf}{\thesection.}{0em}{}%\titlespacing*{\subsection}{0pt}{0\baselineskip}{0\baselineskip}
\renewcommand{\thesection}{\arabic{section}}

\titleformat{\subsection}{\flushleft \normalsize \bf}{\thesubsection}{0em}{}
\renewcommand{\thesubsection}{\arabic{section}.\arabic{subsection}}

%No indent
\setlength\parindent{24pt}
\setlength{\parskip}{5pt}

%Logo
%\AddToShipoutPictureBG{%
%  \AtPageUpperLeft{\raisebox{-\height}{\includegraphics[width=1.5cm]{uchicago.png}}}
%}

\newcolumntype{L}[1]{>{\raggedright\let\newline\\\arraybackslash\hspace{0pt}}m{#1}}
\newcolumntype{C}[1]{>{\centering\let\newline\\\arraybackslash\hspace{0pt}}m{#1}}
\newcolumntype{R}[1]{>{\raggedleft\let\newline\\\arraybackslash\hspace{0pt}}m{#1}} 

\newcommand{\mr}{\multirow}
\newcommand{\mc}{\multicolumn}

%\newcommand{\comment}[1]{}

\begin{document}


\title{\Large \textbf{The Effect of Opportunity Zone Policy on Technology Entrepreneurs}}
\author{Yi Ling}

\date{\today}

\maketitle

\thispagestyle{empty} 
\doublespacing
\thispagestyle{empty} 

\section{Research Question}
Does Opportunity Zone Policy Provide an Opportunity for Entrepreneurs?

\section{Research Background and Literature}
In recent decades, we have witnessed a decline in entrepreneurship, including a decrease in entry rates and small, skilled firms, which will lead to a decrease in labor demand and lower productivity \citep{jiangSkillbiasedEntrepreneurialDecline2023}. High entry cost from financial frictions will decrease the entry of highly productive firms and technology adoption, eventually leading to high losses from Total Factor Productivity (TFP) \citep{midrigan2014finance}. 

To address this problem, policies that lower the cost of starting new firms can facilitate the entry of skilled entrepreneurs.  In recent years, many policies have helped to support the entry of start-ups, including the Paycheck Protection Program, Small Business Data Collection Rule, Opportunity Zones in the Tax Cut and Job Act. For instance, \cite{hubbard2020has} studies the effect of the Paycheck Protection Program on protecting employment in small firms. This proposal aims to analyze whether the launch of the Opportunity Zone policy, which provides a location-based lower tax rate for investors, could attract entrepreneurs to start a business, or whether it merely induces firms to relocate from high-tax areas to Opportunity Zones.

The Tax Cuts and Jobs Act of 2017 was launched by the U.S. Congress, stimulating investment in low-income neighborhoods: Opportunity Zones. As a location-based tax incentive, firms building in the opportunity zone received a reduced corporate tax rate, and investors who invest in the firms in the opportunity zone received a reduced capital tax rate. Many researchers discuss the policy's effect on employment, housing prices. For example, \cite{arefeva2020benefits} found the increase in labor supply from the Opportunity Zone program. \cite{kennedy2021neighborhood} document higher income and investments in the Opportunity Zone neighborhood.

\section{Economic Framework and Empirical Design}
A Difference-in-Difference framework is designed to evaluate the effect of the Opportunity Zone policy on the entry of new tech companies. First, at the individual level, the outcome is whether a person residing in an Opportunity Zone becomes an entrepreneur. This measure captures the policy’s effect on entrepreneurial entry at the micro level. Second, at the location level, the outcome is the number of patent applications filed within each zip code. 

\section{Data}
The Longitudinal Employer-Household Dynamics (LEHD) program at the U.S. Census Bureau provides information on people's employment status, wages, and location.  Survey of Business Owners (SBO) also provides information on the type of entrepreneurs. The United States Patent and Trademark Office (USPTO) also helps identify the immigrant investors and patent application location. Some researchers also use the Internal Revenue Service (IRS) to identify the impact from the Opportunity Zone.

\bibliographystyle{apalike} 
\bibliography{references}

\end{document}

